% TEMPLATE for Usenix papers, specifically to meet requirements of
%  USENIX '05
% originally a template for producing IEEE-format articles using LaTeX.
%   written by Matthew Ward, CS Department, Worcester Polytechnic Institute.
% adapted by David Beazley for his excellent SWIG paper in Proceedings,
%   Tcl 96
% turned into a smartass generic template by De Clarke, with thanks to
%   both the above pioneers
% use at your own risk.  Complaints to /dev/null.
% make it two column with no page numbering, default is 10 point

% Munged by Fred Douglis <douglis@research.att.com> 10/97 to separate
% the .sty file from the LaTeX source template, so that people can
% more easily include the .sty file into an existing document.  Also
% changed to more closely follow the style guidelines as represented
% by the Word sample file. 

% Note that since 2010, USENIX does not require endnotes. If you want
% foot of page notes, don't include the endnotes package in the 
% usepackage command, below.

% This version uses the latex2e styles, not the very ancient 2.09 stuff.
\documentclass[letterpaper,twocolumn,10pt]{article}
\usepackage{usenix,epsfig,endnotes,url}

%%% PDFLATEX Letter dimensions
\usepackage[letterpaper]{geometry}
\pdfpagewidth 8.5in
\pdfpageheight 11.0in



\usepackage[HideComments]{comment}

\usepackage[english]{babel}
\usepackage[latin1]{inputenc}
\usepackage{amsmath,amsthm, amssymb, latexsym}
\usepackage{xspace}

\usepackage{graphicx}
\usepackage{multirow}
\usepackage{booktabs}

\newtheorem{researchquestion}{RQ}
\newcommand{\Hz}{\ensuremath{\mathrm{Hz}\space}}
\newcommand{\const}[1]{\ensuremath{\mathsf{#1}\xspace}}
\newcommand{\vari}[1]{\ensuremath{\mathit{#1}\xspace}}


\begin{document}

%don't want date printed
\date{}




%make title bold and 14 pt font (Latex default is non-bold, 16 pt)
\title{\Large \bf Does simulating a serious information security threat cause a user's self-efficacy to increase, with regards to the fear appeals of Protection Motivation Theory? }

%for single author (just remove % characters)
\author{
Mark Shaw\\
Newcastle University
%}\\
%thomas.gross@newcastle.ac.uk
%\and
%School of Computing Science\\
%Newcastle University\\
%Newcastle-upon-Tyne, NE1 7RU, UK\\
% copy the following lines to add more authors
% \and
% {\rm Name}\\
%Name Institution
} % end author

\maketitle

% Use the following at camera-ready time to suppress page numbers.
% Comment it out when you first submit the paper for review.
%\thispagestyle{empty}

%\begin{abstract}
%Template for structured abstract according to LASER guidelines
%\end{abstract}

\section{Structured Abstract}
 \label{sec:abstract}

\paragraph{Background.} 
  % State the background and context of the work described in the paper.
Protection Motivation Theory (PMT) and the Extended Parallel Process Model (EPPM) have been used to demonstrate that fear appeals can perform a dutiful role in motivating users to increase their self-efficacy regarding Information Security (ISec) concerns. Currently, there is very little research into whether having a user be confronted with a threat that they perceive to be real, the realisation that the threat is not threatening should allow for users to increase their perceived susceptibility of ISec threats, thus allowing users to increase their self-efficacy. 

\paragraph{Aim.} 
  % State the research question, objective, or purpose of the work in the paper.
Does simulating a ransomware attack on a user increase the user's self-efficacy with regards to Information Security?

\paragraph{Method.}
  % Briefly summarize the method used to conduct the research, including the subjects, procedure, data, and analytical method.

A blind study was carried out on two groups of 
$N$ subjects. During the study, one group experienced a simulated ransomware attack, the other did not. A questionnaire was then completed by the groups, and ISec concerns were measured and compared across groups.


\paragraph{Anticipated Results.}
  % State the outcome of the research using measures appropriate for the study conducted. Results are essentially the numbers.
We anticipate that the users who experience the simulated ransomware attack will show greater concern for information security, compared to the control condition. 

\paragraph{Anticipated Conclusions.}
  %  State the lessons learned as a result of the study and recommendations for future work. The conclusions are the ``so what'' of the study.
  %  Impact
The Limited Parallel Process Model predicts that users who experience a threat have heightened levels of perceived susceptibility, which should in turn allow for these users to have raised self-efficacy. The study will investigate whether the relief that comes from the realisation that the threat was simulated is enough to positively enforce self-efficacy.

%Hence, we expect the quality of the results of the decision, here the quality of passwords, to deteriorate, yielding more human-made faults impacting security.
\?{So what? What would be a better task than password generation?}

%\section{Introduction}
%Lit review

%So what: if we find that memorability and strength of passwords are poorer when people are depleted - maybe it means that a key security guideline is to ensure a non-depleted state when choosing a password. Now is this enough of a deal breaker?

%
%
%\section{Method}
%Template: Between-subject design, 1 IV, 2 levels, 2 DVs.\tgr{Not understood by Computer Science community?}
%
%% Subjects
%Two groups of $N$ subjects are chosen ($N$ determined in from a pre-test on effect size observed, for now fix 30 in experiment condition, 30 in control, as rule of thumb).
%(Requirements of sample being representative? Of what population?) The subjects are designated to be Joe Users\?{How do we ensure that a user is a Joe User?!}.
%\?{What is the effect size? How large a sample to we need? (computed from effect size)}
%
%% Procedure
%We structure the procedure as follows:
%\emph{Pre-Experiment.} (what do we tell subjects and how?) The subjects are informed about the experiment and the measurement procedure (eye tracking, calibration, physiological measurement). 
%The subjects given a pre-experiment questionnaire incl. demographics and CS/security expertise (determine differences from expertise), and self-report on pre-experiment depletion (cf. Baumeister's format).
%\?{What are actual questionnaire components to use? (check Baumeister's work)}
%
%\emph{Experiment.} 
%The Independent Variable (IV) is cognitive depletion with two levels: depleted (experiment condition), non-depleted (control condition).
%The Dependent Variables (DVs) are password strength (measured with Password Meter\footnote{\url{http://www.passwordmeter.com/}}, the Microsoft Password checker\footnote{\url{http://www.microsoft.com/en-gb/security/pc-security/password-checker.aspx}} or entropy)\?{What are the best password strength measures? Which are usable for the study in automated evaluation?} and password memorability (measured by a psychological metric to be determined in collaboration with Pam Briggs or security results, e.g., Jeff Yan Password memorability and security: Empirical results, Oakland 2014).\?{How to actually measure password memorability? What is a good metric, computable automatically for the input observed?}
%
%The IV is manipulated with depletion tasks as established in psychology (cf. Baumeister, Kahnemann), primary candidate Baumeister's Attention Refocusing task. The IV is controlled with physiological measurements, (a) Task-Evoked Pupiliary Response (Measurement device: RED500 Eye Tracker), (b) Event-Related Skin Conductance Response (ER-SCR) and Skin Conductance Level (SCL) (Measurement device: Biopac EDA suite), (c) heart-rate variability (Measurement device: Biopac ECG).
%\?{How do we manipulate the depletion? Are the depletion tasks proposed reliable?}
%\?{\textbf{PRE-TEST 1:} Baumeister Attention Refocusing Depletion task, with control of through measurement apparatus.}
%\?{\textbf{PRE-TEST 2:} Depletion as balanced and non-balanced task. Difference?}
%
%
%\?{What is the recovery time after the depletion manipulation of the IV? (cf. Baumeister)}
%
%\begin{enumerate}
%  \item Both experiment condition and control are set in a neutral control state (e.g., listening to music), to eliminate pre-experiment arousal/stress.\?{Do we need a neutral state at all? Why? Why not?}
%  \item We use Baumeister's Attention Refocusing Task to induce depletion, where experiment condition and control condition get to see the same video for a defined time.
%   The experiment condition received the instruction to refocus attention on the centre of the video, the control condition is instructed to only watch the video. (According to Baumeister this task is depleting for the experiment condition, and non-depleting for the control condition.
%  \item Experiment and control condition both are requested to chose a password for a new GMail account.
%\end{enumerate}
%
%\emph{Post-Experiment.}
%The subjects are given a post-study questionnaire incl. self-report on post-experiment depletion (cf. Baumeister) and NASA-TLX (as alternative self-reported measure of perceived task load.
%
%% Data
%\emph{Data.} As data we will have $N$ datasets of 
%\begin{enumerate}
%  \item pre-and post-experiment questionnaires, 
%  \item physiological data on the IV control, RED500: pupil diameter time series, also available blink rate etc., Biopac EDA: SCR and SCL, Biopac ECG, all time-synchronized with Observer XT. \?{How do we use the physiological data to evaluate that the depletion IV was properly controlled?}
%  \item Password input
%\end{enumerate}
%
%\?{\textbf{Open question to check:} Do we need $m$ repetitions?}
%
%
%% Analytical Method
%\emph{Analytical Method.}





\bibliographystyle{abbrv}
\bibliography{laser}



\end{document}


































